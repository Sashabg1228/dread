\begingroup % Begin a TeX group in order to localize scope of next three instructions
    \small
    \begin{longtable}{| m{4.5cm} | m{4.5cm} | m{3cm} | m{2cm} |}
    %% Headers and footers
    \caption{Използвани елементи в печатната платка на робота}
    \label{table:elements-robot} \\
    \hline
    \textbf{Вид} 
      & \textbf{Продуктов номер / Размер}
      & \textbf{Reference(s)}
      & \textbf{брой} \\ \hline
    \endfirsthead 
    \hline
    \textbf{Вид} 
      & \textbf{Продуктов номер / Размер}
      & \textbf{Reference(s)}
      & \textbf{брой} \\ \hline
    \endhead 
    \multicolumn{4}{r@{}}{\em}\\
    \endfoot
    \endlastfoot
    %% Body of table
    Електролитен Кондензатор & 100uF & C1, C5, C8 & 3 
    \\ \hline
    Керамичен Кондензатор & 100nF & C2, C3, C4, C6, C7, C9, C10, C11, C12 & 9
    \\ \hline
    Ценеров диод & BZX85C10-TAP & D1 & 1
    \\ \hline
    Двупосочен трансил & P6KE27CA & D3 & 1
    \\ \hline
    Оптрони & HCPL-2531 & IC1, IC2, IC3 & 3
    \\ \hline
    Клема 2 пина & 236-102 & J1, J3, J4, J7, J16, J6, J9, J18, J13, J14, J15 & 11
    \\ \hline
    Клема 4 пина & 236-104 & J2, J10, J11 & 1
    \\ \hline
    Клема 1 пин & 236-401 & J5, J8, J12, J17 & 4
    \\ \hline
    Бобина & 2.2uH & L1 & 1
    \\ \hline
    P-MOS транзистор & IXTP120P065T & Q1 & 1
    \\ \hline
    Резистор & 8.2kR & R1 & 1
    \\ \hline
    Резистор & 680R & R2, R3, R4 & 3
    \\ \hline
    Линеен регулатор & L7805 & U3, U4 & 2
    \\ \hline
    Преобразувател на логическото ниво & CD40109BE & U12 & 1
    \\ \hline
    Гнездо за бушон & 3522-2 & X1 & 1
    \\ \hline
    \end{longtable}
\endgroup % End of TeX group


%================================================================================
%================================================================================


\begingroup % Begin a TeX group in order to localize scope of next three instructions
\begin{landscape}
    \small
    \begin{longtable}{| m{2cm} | m{5cm} | m{3cm} | m{3cm} | m{9cm} |}
    %% Headers and footers
    \caption{Конфигурация на изводите на микроконтролера на робота}
    \label{table:pins-robot} \\
    \hline
    \textbf{Pin Nо.} 
      & \textbf{User label}
      & \textbf{GPIO mode}
      & \textbf{GPIO pull-up}
      & \textbf{GPIO function} \\ \hline
    \endfirsthead 
    \hline
    \textbf{Pin Nо.} 
      & \textbf{User label}
      & \textbf{GPIO mode}
      & \textbf{GPIO pull-up}
      & \textbf{GPIO function} \\ \hline
    \endhead 
    \multicolumn{5}{r@{}}{\em Cont'd on following page}\\
    \endfoot
    \endlastfoot
    %% Body of table
    PA1 & HAND\_PWM & GPIO\_OUTPUT & NO & Чрез таймер 2 се генерират импулси за управление на стъпковия мотор\\
            \hline
            PA2 & HAND\_DIR & GPIO\_OUTPUT & NO & Управлява се посоката на движение на стъпковия мотор\\
            \hline
            PA4 & DISK\_DIR & GPIO\_OUTPUT & NO & Управлява се посоката на движение на мотора за оръжието\\
            \hline
            PA5 & WHEEL\_LEFT\_DIR & GPIO\_OUTPUT & NO & Управлява се посоката на движение на мотора за лявото колело\\
            \hline
            PA6 & WHEEL\_RIGHT\_DIR & GPIO\_OUTPUT & NO & Управлява се посоката на движение на мотора за дясното колело\\
            \hline
            PA7 & DISK\_PWM & TIM3\_CH2 & NO & Чрез канал 2 на таймер 3 се генерира ШИМ сигнал за управление на скоростта на мотора за оръжието\\
            \hline
            PB0 & WHEEL\_LEFT\_PWM & TIM3\_CH3 & NO & Чрез канал 3 на таймер 3 се генерира ШИМ сигнал за управление на скоростта на мотора за лявото колело\\
            \hline
            PB1 & WHEEL\_RIGHT\_PWM & TIM3\_CH4 & NO & Чрез канал 4 на таймер 3 се генерира ШИМ сигнал за управление на скоростта на мотора за дясното колело\\
            \hline
            PB14 & HAND\_END\_LEFT & GPIO\_EXTI14 & PULL-UP & Чрез външното прекъсване на този пин, генерирано при падащ и растящ фронт, се предотвратява излизането на стъпковия мотор извън крайното си състояние\\
            \hline
            PB15 & HAND\_END\_RIGHT & GPIO\_EXTI15 & PULL-UP & Чрез външното прекъсване на този пин, генерирано при падащ и растящ фронт, се предотвратява излизането на стъпковия мотор извън крайното си състояние\\
            \hline
             PB7 & OPT\_END\_WHEEL\_RIGHT & GPIO\_EXTI7 & PULL-UP & Чрез външното прекъсване на този пин, генерирано при падащ фронт, се проследява каква е скоростта на мотора за дясното колело\\
            \hline
            PB8 & OPT\_END\_WHEEL\_LEFT & GPIO\_EXTI8 & PULL-UP & Чрез външното прекъсване на този пин, генерирано при падащ фронт, се проследява каква е скоростта на мотора за лявото колело\\
            \hline
            PB9 & OPT\_END\_WHEEL\_DISK & GPIO\_EXTI9 & PULL-UP & Чрез външното прекъсване на този пин, генерирано при падащ фронт, се проследява каква е скоростта на мотора за оръжието\\
            \hline
            PB5 & SPI1\_MOSI & SPI1\_MOSI & NO & Използва се за MOSI пин при помуникацията с радиочестотния модул\\
            \hline
            PB4 & SPI1\_MISO & SPI1\_MISO & NO & Използва се за MISO пин при помуникацията с радиочестотния модул\\
            \hline
            PB3 & SPI1\_SCK & SPI1\_SCK & NO & Използва се за SCK пин при помуникацията с радиочестотния модул\\
            \hline
            PA10 & NRF24L01\_IRQ & GPIO\_EXTI10 & NO & Чрез външното прекъсване на този пин, генерирано при падащ фронт, се известява за това, че има получен пакет информация\\
            \hline
            PA11 & NRF24L01\_CE & GPIO\_OUTPUT & NO & Управлява се дали може да се изпращат и получават данни\\
            \hline
            PA15 & NRF24L01\_CSN & GPIO\_OUTPUT & NO & Управлява се дали може да се четат и пишат регистрите на радиочестотния модул\\
            \hline
    \end{longtable}
\end{landscape}
\endgroup % End of TeX group
