\chapter{Заключение}

\section{Постигнати резултати}

Успешно е реализирано проектирането и изработката на механичното конструкция на бойния робот. Осъществен е и контрола на движението на постояннотоковите четкови мотори посредством широчинно-импулсен сигнал. Инсталирани са и са тествани оптични сензори, с които се мери скоростта на четковите мотори. Реализирано е и управлението на стъпковия мотор на механичната ръка. Инсталирани са и крайните изключватели, които да следят да не се преминават крайните позиции от ръката. Постигната е и безжичната радиовръзка робот - пулт за управление. По време на проектирането на дипломния проект бяха спазени всички изисквания от правила на Tues BattleBots 2024 и дефинираните функции за безопасност.

\section{Бъдещо развитие}

Основните недостатъци на разработения дипломен проект са свързани с избрания радиочестотен модул и неговата употреба в проекта. Избраният модул въпреки, че има възможността да предава и получава повече от достатъчно количество информация на голямо разстояние, не може да го прави едновременно. За да се постигне двупосочна комуникация трябва постоянно да се преконфигурира неговият режим на работа. 

Разработената безжична комуникация е постигната без употребата на криптиране на инструкциите към робота. Този факт излага на риск хората и техниката около робота поради причината, че се дава възможност на злонамерени лица да управляват робота.

Към момента на предаване на дипломния проект има реализирана функция за отмерване на скоростта на постояннотоковите мотори, но тази информация е достъпна само за микроконтролера в робота. На пулта за управление може да бъде добавен дисплей и да се разработи графичен интерфейс, който да показва отмерените скорости, състоянието на оръжието и друга информация като нивото на батерията на робота.

