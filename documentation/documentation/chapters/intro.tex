\chapter{Увод}
Още от зората си, човечеството търси разнообразни и вълнуващи начини да успее да избяга от еднообразното си ежедневие. За целта много и различни методи за развлечение са възникнали, вариращи от приятелски игри до зрелищни спектакли. Едно от най-разпространените забавления още от тогава е също толкова популярно и днес, а именно – боевете. Зрелището на това два индивида да се бият помежду си завладява всеки зрител. Годините са доказали, че колкото по-драматична е една битка, толкова по-силни са чувствата, които се пораждат у нейните наблюдатели. Но това води до един неизбежен проблем: участниците в такива интензивни битки винаги биват физически наранени. Поради тази причина се появява казусът как може да се постигнат тези зрелища, без участниците да пострадат. Решението на този проблем са битките с бойни роботи (или батълботи). Стандартна битка трае 3 минути и в тях роботизираните системи, контролирани с дистанционно управление, целят увреждането на опонента до такава степен, че вече да не може да извършва движения.