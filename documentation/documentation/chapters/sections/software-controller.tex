\section{Софтуерна реализация на дистанционното}
\label{sec:software-controller}

Както e описано в \cref{sec:block-schemas} компонентите, които се използват в дистанционното са 2 потенциометъра и 3 бутона.

%================================================================================
% ИНИЦИАЛИЗАЦИЯ

\subsection{Инициализация на микроконтролера на дистанционното}
\label{ssec:init-controller}

След като получи захранването си микроконтролера преминава през серия от процеси, които трябва да конфигурират използваните периферии, изводи и променливи. Посредством функцията HAL\_Init() се рестартират всички периферии и се инициализират HAL библиотеката и всички нейни функционалности. След това чрез SystemClock\_Config() се конфигурира системния таймер. После следва конфигурацията на изводите на микроконтролера. Една от тях може да бъде видяна в \autoref{lst:conf-pin-controller}. За това кой пин как е конфигуриран и за какво се използва може да се разбере от таблица \cref{table:pins-controller}. След това се инициализира SPI връзката с nRF модула. Чрез MX\_TIM4\_Init() се инициализира таймер 4, с помощта на когото се изпращат пакетите с инструкции на равни периоди. След това се инициализират и двата аналогово-цифрови преобразувателя, които се използват за четене на състоянието на потенциометрите. Последната инициализация е тази на радиочестотния модул и се задава неговия режим на работа. Преди да започне повтарящият се цикъл на работа таймера се стартира и се пуска калибрация на аналогово-цифровите преобразуватели.

\lstinputlisting[language=c, consecutivenumbers=false, linerange={72-76}, caption={Конфигуриране на пин}, label={lst:conf-pin-controller}]{documents/controller/gpio.c}

\begin{table}[H]
    \centering
    \begin{tabular}{| m{4cm} | m{3,5cm} | m{3,5cm} | m{3,5cm} |}
        \hline
        & Wi-fi & Bluetooth & nRF24L01+ \\
        \hline
        Скорост &  Висока & Средно & Средна \\
        \hline
        Обхват & 10ки метри & 10 метра & 10-150 метра \\
        \hline
        Енергийна консумация & Висока & Средна & Ниска\\
        \hline
    \end{tabular}
    \caption{Пинове}
    \label{table:pins-controller}
\end{table}

%================================================================================
% РАБОТЕН ЦИКЪЛ

\subsection{Работен цикъл на дистанционното}
\label{ssec:loop-controller}

Както е описано в \cref{sec:logic-schemas} цикъла на работа се повтаря на равни интервали. Това е постигнато посредством прекъсването при преливане на таймер 4. Той е конфигуриран така, че прекъсването да бъде генерирано на всяка десета от секундата. След извикването на прекъсването се вдига флага send\_flag. При това действие се прочитат състоянията на потенциометрите и се записват на полетата за скорост на лявото и дясното колело. Това е последвано от извикването на функцията за калибриране на данните. В края на този цикъл се инициира изпращане на пакета с инструкциите и при успех се сменя състоянието на вградения светодиод.

\lstinputlisting[language=c, consecutivenumbers=false, linerange={113-134}, caption={Работен цикъл}, label={lst:loop-controller}]{documents/controller/controller.c}

Успоредно на това има конфигурирани и външни прекъсвания. Те имат функцията да редактират позицията на ръката и да пускат или спират движението на оръжието.


